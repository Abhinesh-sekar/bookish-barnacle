\documentclass{article}
\usepackage{darkmode}

\title{Math Course Notes}
\author{Your Name}
\date{\today}

\begin{document}

\maketitle

\section*{Real Analysis}
\subsection*{Limits and Continuity}
\begin{mydefinition}{Definition}
A function \( f: \R \to \R \) is said to be \textit{continuous} at a point \( c \in \R \) if for every \( \epsilon > 0 \), there exists \( \delta > 0 \) such that whenever \( |x - c| < \delta \), it follows that \( |f(x) - f(c)| < \epsilon \).
\end{mydefinition}

\subsection*{Differentiation}
\begin{mytheorem}{Theorem}
If \( f \) is differentiable at \( c \in \R \), then \( f \) is continuous at \( c \).
\end{mytheorem}
\begin{myproof}
Let \( \epsilon > 0 \) be given. Since \( f \) is differentiable at \( c \), there exists \( \delta > 0 \) such that for all \( x \) with \( 0 < |x - c| < \delta \), we have
\[
\left| \frac{f(x) - f(c)}{x - c} - f'(c) \right| < \frac{\epsilon}{|x - c|}
\]
Thus,
\[
|f(x) - f(c)| < \epsilon
\]
which shows that \( f \) is continuous at \( c \).
\end{myproof}

\subsection*{Examples}
\begin{myexample}{Example 1}
Consider the function \( f(x) = x^2 \). This function is continuous at every point \( c \in \R \). To see this, note that
\[
|f(x) - f(c)| = |x^2 - c^2| = |x - c||x + c|
\]
Since \( |x + c| \) is bounded near \( c \), \( |f(x) - f(c)| \) can be made arbitrarily small by choosing \( x \) sufficiently close to \( c \).
\end{myexample}

\end{document}